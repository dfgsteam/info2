% Dokoumentenklassen legen das grobe Layout fest. Hier wird eine selbsterstellte Klasse verwendet, deswegen benötigt man zum erstellen des PDFs auch die Datei uebungsblatt.cls und uebungsblatt.sty
\documentclass{uebungsblatt}

\usepackage[linesnumbered,commentsnumbered]{algorithm2e}
\usepackage{tikz}
\usepackage{hyperref}
% Store \nl in \oldnl
\let\oldnl\nl
% Remove line number for one line
\newcommand{\nonl}{\renewcommand{\nl}{\let\nl\oldnl}}

%Hier wird die Kopfzeile erstellt
\header{\textbf{Grundlagen der Praktischen Informatik}\hfill\textbf{Sommersemester 2024}\\
Student:in1 (Matrikelnummer1) % Name des Gruppenmitglieds Eintragen
\hfill Georg-August-Universität Göttingen \\
Student:in2 (Matrikelnummer2)  % Name des Gruppenmitglieds Eintragen
\hfill Institut für Informatik\\ 
Student:in3 (Matrikelnummer2)\\  % Name des Gruppenmitglieds Eintragen
Student:in4 (Matrikelnummer2)\\  % Name des Gruppenmitglieds Eintragen

% Hiermit wird der horizontale Strick erzeugt, der die Kopfzeile abgrenzt.
\rule{\textwidth}{0.1mm}}

% Hier wird die Blattnummer festgelegt.
\blattnummer{5}

% Hier beginnt der eigentliche Textkörper. Alles zwischen \begin{document} und \end{document} ist der eigentliche Text
\begin{document}

% Mit \underline kann man Sachen unterstreichen


% \begin{aufgabe} erstellt ein Aufgabenumgebung mit [...] könnt ihr den Titel angeben und mit \score die Punktzahl festlegen. Ist aber für euch nicht so wichtig.
\begin{aufgabe}[Pumping Lemma \score{17}]
Zeigen Sie, dass die Sprache 
$L = \{a^p \ | \ p \in \mathbb{N} \text{ ist eine Primzahl}\}$ nicht regulär ist.\\
\score{17}

\medskip
\underline{Hinweise}

\begin{itemize}
\item
Es gibt unendlich viele Primzahlen, d.h. für jedes 
$k \in \mathbb{N}$ gibt es eine Primzahl $p \geq k + 2$.
\item
Der Widerspruch $xy^iz \not\in L$, d.h. $|xy^iz|$ ist keine Primzahl,
muss mit einem passend gewählten $i \in \mathbb{N}$ geführt werden.
\end{itemize}

\end{aufgabe}
% Hier könnt ihr eure Lösung hinschreiben
\begin{loesung} 

\end{loesung}
\newpage
%-----------------------------
\begin{aufgabe}[Grammatik \score{26}]
\medskip
Gegeben sei folgende Sprache über dem Alphabet $\Sigma = \{a,b\}$.
\begin{align*}
L = \{ w \in \Sigma^* \ | \ w\ \mbox{enthält weder das Teilwort $aa$ noch das Teilwort $bb$} \}
\end{align*}

\begin{enumerate}
\item
\label{kurzeWoerter}
Geben Sie die sieben kürzesten Wörter aus $L$ an.\\
\score{4}
\item
\label{abGrammatik}
Geben Sie eine reguläre Grammatik $G$ an, die $L$ erzeugt.
Benutzen Sie dabei höchstens 4 Nichterminale.\\
\score{8}
\item
Zeigen Sie für die beiden längsten Wörter aus Aufgabenteil~\ref{kurzeWoerter}., jeweils
durch Angabe einer Ableitung, das diese Wörter zur Sprache $L(G)$ gehören, die
von der Grammatik aus Aufgabenteil \ref{abGrammatik}. erzeugt wird.\\
\score{4}
\item
Ist $L$ eine reguläre Sprache? Mit Begründung.\\
\score{2}
\item
Legen Sie eine natürliche Zahl $k \in \mathbb{N}$ fest und 
geben Sie für jedes Wort $w \in L$ mit $|w| \ge k$ eine Zerlegung 
$xyz \in \{a, b\}^*$ an, für die gilt
\begin{itemize}
\item
$w = xyz$,
\item
$|y| \ge 1$, 
\item
$|xy| \le k$,
\item
für alle $i \in \mathbb{N} \cup \{0\}$ gilt, $xy^{i}z \in L$.
\end{itemize}
\score{8}
\end{enumerate}
\end{aufgabe}
\begin{loesung}

\end{loesung}
\newpage
\begin{aufgabe}[Rechts-/Linkslineare Grammatik \score{16}]
\medskip
Gegeben sei folgende rechtslineare Grammatik $G = (N,T,P,S)$.
\begin{itemize}
\item
Nichtterminale
$N := \{$ START, BIN, NULL, OP $\}$
\item
Terminale
$T := \{0, 1, \vee, \wedge\}$
\item
Produktionen 
\begin{tabbing}
$P$ := \{ \quad \= FACTOR \quad \= \kill
$P$ := \{ \> START  \>$\rightarrow$ 1 BIN $|$ 0 NULL $|$ 1 $|$ 0  \\
          \> BIN    \>$\rightarrow$ 1 BIN $|$ 0 BIN $|$ $\vee$ OP $|$ $\wedge$ OP $|$ $\varepsilon$ \\
          \> NULL   \>$\rightarrow$ $\vee$ OP $|$ $\wedge$ OP $|$ $\varepsilon$ \\
          \> OP     \>$\rightarrow$ 1 BIN $|$ 0 NULL \quad \}
\end{tabbing}
\item
Startsymbol $S :=$ START
\end{itemize}

\begin{enumerate}
\item
\label{zweiWoerter}
Geben Sie zwei Worte der von $G$ erzeugten Sprache $L(G)$ an, 
die jeweils mit $0$ und $1$ beginnen,
jeweils jedes Terminalsymbol mindestens einmal enthalten und
insgesamt keine Ziffernfolge mehr als einmal enthalten.\\
\score{2}
\item
\label{linksGrammatik}
Geben Sie eine linkslineare Grammatik $G'$ an, 
die dieselbe Sprache wie die rechtslineare Grammatik $G$
erzeugt, d.h. es gilt $L(G') = L(G)$.\\
\score{10}
\item
Zeigen Sie für die beiden Wörter aus Aufgabenteil~\ref{zweiWoerter}., jeweils
durch Angabe einer Ableitung, das diese Wörter zur Sprache $L(G')$ gehören, die
von der Grammatik aus Aufgabenteil \ref{linksGrammatik}. erzeugt wird.\\
\score{4}

\end{enumerate}
\end{aufgabe}
% Hier könnt ihr eure Lösung hinschreiben
\begin{loesung}

\end{loesung}
\end{document}
