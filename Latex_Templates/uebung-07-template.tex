% Dokoumentenklassen legen das grobe Layout fest. Hier wird eine selbsterstellte Klasse verwendet, deswegen benötigt man zum erstellen des PDFs auch die Datei uebungsblatt.cls und uebungsblatt.sty
\documentclass{uebungsblatt}

\usepackage[linesnumbered,commentsnumbered]{algorithm2e}
\usepackage{tikz}
\usepackage{hyperref}
% Store \nl in \oldnl
\let\oldnl\nl
% Remove line number for one line
\newcommand{\nonl}{\renewcommand{\nl}{\let\nl\oldnl}}

%Hier wird die Kopfzeile erstellt
\header{\textbf{Grundlagen der Praktischen Informatik}\hfill\textbf{Sommersemester 2024}\\
Student:in1 (Matrikelnummer1) % Name des Gruppenmitglieds Eintragen
\hfill Georg-August-Universität Göttingen \\
Student:in2 (Matrikelnummer2)  % Name des Gruppenmitglieds Eintragen
\hfill Institut für Informatik\\ 
Student:in3 (Matrikelnummer2)\\  % Name des Gruppenmitglieds Eintragen
Student:in4 (Matrikelnummer2)\\  % Name des Gruppenmitglieds Eintragen

% Hiermit wird der horizontale Strick erzeugt, der die Kopfzeile abgrenzt.
\rule{\textwidth}{0.1mm}}

% Hier wird die Blattnummer festgelegt.
\blattnummer{7}

% Hier beginnt der eigentliche Textkörper. Alles zwischen \begin{document} und \end{document} ist der eigentliche Text
\begin{document}

% Mit \underline kann man Sachen unterstreichen


% \begin{aufgabe} erstellt ein Aufgabenumgebung mit [...] könnt ihr den Titel angeben und mit \score die Punktzahl festlegen. Ist aber für euch nicht so wichtig.
\begin{aufgabe}[KNF \score{8}]
Gegeben sei die folgende aussagenlogische Formel $F$ mit den atomaren Aussagen $A, B, C, D$.

\begin{align*}
\bigl((C \land \neg D) \land \neg (\neg D \implies B)\bigr) 
\lor 
\bigl((\neg A \lor \neg B) \land (\neg A \implies \neg C)\bigr)
\end{align*}

Formen Sie die Formel in Konjunktive Normalform (KNF) um und geben Sie die resultierende Klauselmenge an.\\
\score{8} 


\end{aufgabe}
% Hier könnt ihr eure Lösung hinschreiben
\begin{loesung} 

\end{loesung}
\newpage
%-----------------------------
\begin{aufgabe}[Resolutionskalkül \score{13}]
Betrachten Sie die folgende aussagenlogische Formel.
\begin{align*}
F = 
(X\vee \neg Y) \wedge
(\neg X\vee \neg Y\vee \neg Z) \wedge 
(\neg X\vee Z) \wedge 
(X\vee Y\vee Z) \wedge
(Y\vee \neg Z)
\end{align*}

\begin{enumerate}
\item
Bestimmen Sie die Klauselmenge $\mathcal{K}_0$ der Formel $F$.\\
\score{2}
\item
F ist nicht erfüllbar. Zeigen Sie das mit dem Resolutionskalkül
der Aussagenlogik.

Geben Sie für jedem Schritt $i$ folgendes an.
\begin{align*}
\mathcal{K}_i = \mathrm{Res}(\mathcal{K}_{i-1}) = \mathcal{K}_{i-1} \cup \{R\ |\ R \mbox{ ist Resolvente von } K', K'' \in \mathcal{K}_{i-1} \} 
\end{align*}
\score{11}

\underline{Hinweis}

Der letzte Schritt ist die Bildung der Resolvente der
Klauseln $\{Z\}$ und $\{\neg Z\}$. 

\smallskip
\underline{Beispiel aus dem Skript}
\begin{align*}
\mathcal{K}_0 &= \Bigl\{\{P, S\},\{\neg P, \neg S\},\{\neg S, A\},\{\neg A, P\},\{P\},\{S\}\Bigr\} \\
\mathcal{K}_1 &= Res(\mathcal{K}_0) = \mathcal{K}_0 \cup \Bigl\{ \{\neg S\}\ | 
\mbox{ Resolvente von } \{\neg P, \neg S\}, \{P\} \Bigr\} \\
\mathcal{K}_2 &= Res(\mathcal{K}_1) = \mathcal{K}_1 \cup \Bigl\{ \quad \emptyset\quad\, | 
\mbox{ Resolvente von } \{\neg S\}, \{S\} \Bigr\} 
\end{align*}


\end{enumerate}

\end{aufgabe}
\begin{loesung}

\end{loesung}
\newpage
\begin{aufgabe}[Formeln und Terme \score{18}]
Betrachten Sie eine einfache arithmetische Sprache
über der Individuenmenge der ganzen Zahlen $\mathbb{Z}$
mit folgender Signatur.
Die arithmetischen Operationen und die Größenrelationen über $\mathbb{Z}$ werden
als bekannt vorausgesetzt.

\begin{itemize}
\item
Prädikate $\mathcal{P} = \{ <, > \}$.
\begin{itemize}
\item Alle Prädikate habe die Stelligkeit 2.
\item Es gibt $X < Y$ ist genau dann \textit{wahr}, wenn $X$ echt kleiner als $Y$ ist.
\item Es gibt $X > Y$ ist genau dann \textit{wahr}, wenn $X$ echt größer als $Y$ ist.
\end{itemize}
\item
Funktoren $\mathcal{F} = \{ +, - \}$.\begin{itemize}
\item Alle Funktoren habe die Stelligkeit 2.
\item Der Funktor $+$ entspricht der Addition in $\mathbb{Z}$.
\item Der Funktor $-$ entspricht der Subtraktion in $\mathbb{Z}$.
\end{itemize}
\end{itemize}

\medskip
Welche der folgenden Ausdrücke sind Formeln, welche Terme und welche gehören nicht 
zur Sprache? Begründen Sie jeweils Ihre Entscheidung.

\medskip
Kann für einen Formel eine Wahrheitswert angegeben werden, gegeben Sie diesen an 
und begründen Sie Ihre Antwort. 

\medskip
Kann für eine Term das bezeichnete Objekt angegeben werden, geben Sie dieses an 
und begründen Sie Ihre Antwort. 

\begin{enumerate} %%[a)]
\item    
$X + ( -Y)$
\item  
$1+2-X-Y$
\item  
$2 < \bigl( ( 2 > 3) \vee (3 \ge 2) \bigr)$
\item  
$1 > \bigl( \exists X: (X < 0) \bigr)$
\item 
$2 + 3 = 5$
\item 
$(X > 2) \wedge (X + 2)$
\item 
$(Z > 2) \vee \bigl( (Z - 1) < 0 \bigr)$
\item 
$ \forall X: \bigl( X \Rightarrow \bigl( X \cdot (2 + 3) \bigr) \bigr)$
\item 
$\exists X: \Bigl( \exists Y: \bigl( (X > 2) \wedge \bigl( \neg (Y < 3) \bigr) \bigr) \Bigr)$
\item
$6 + 1 - 8$
\item 
$ \forall X: \bigl( \bigl( X \wedge (2 > 3) \bigr) \Rightarrow X \bigr)$
\item 
$\exists : \bigl( (Z > 2) \vee \bigl( (Z - 1) < 0 \bigr) \bigr)$
\end{enumerate}
\score{18}

\end{aufgabe}
% Hier könnt ihr eure Lösung hinschreiben
\begin{loesung}

\end{loesung}
\newpage
\begin{aufgabe}[Sprachen und Formeln \score{36}]
\begin{enumerate}
\item
Gegeben ist eine Sprache
über der Individuenmenge der rationalen Zahlen $\mathbb{Q}$
mit folgender Signatur.
\begin{itemize}
\item
Prädikate $\mathcal{P} = \{ = \}$.

Das Prädikat $=$ hat die Stelligkeit 2 und es gilt
$X = Y$ ist genau dann \textit{wahr}, wenn $X$ und $Y$ das gleiche Objekt aus $\mathbb{Q}$ bezeichnen.

\item
Funktoren $\mathcal{F} = \emptyset$.
\end{itemize}

Geben Sie den Wahrheitswert der folgenden Formeln an und
begründen Sie jeweils ihre Antwort.

\begin{enumerate}
\item 
$\forall X : \bigl( \exists Y : (X = Y) \bigr)$\\
\score{4}

\item 
$\forall X : \bigl( \forall Y : (X = Y) \bigr)$\\
\score{4}

\item 
$\exists X : \bigl( \forall Y : (X = Y) \bigr)$\\
\score{4}

\item 
$\exists X : \bigl( \exists Y : (X = Y) \bigr)$\\
\score{4}

\end{enumerate}

%-----------------------------
\item 
Gegeben ist eine Sprache
über der Individuenmenge der natürlichen Zahlen (inklusive 0)
$\mathbb{N}_0 = \mathbb{N} \cup \{0\}$ 
mit folgender Signatur. Die arithmetischen Operationen über $\mathbb{N}_0$
sind wohlbekannt.

\begin{itemize}
\item
Prädikate $\mathcal{P} = \{ = \}$.

Das Prädikat $=$ hat die Stelligkeit 2 und es gilt
$X = Y$ ist genau dann \textit{wahr}, wenn $X$ und $Y$ das gleiche Objekt aus $\mathbb{N}_0$ bezeichnen.

\item
Funktoren $\mathcal{F} = \{ +, dreieck, quadrat \}$.

Der Funktor $+$ hat die Stelligkeit 2 und entspricht der Addition in $\mathbb{N}_0$.

Der Funktor $dreieck$ hat die Stelligkeit 1. Es gilt 
$dreieck(0) = 0$ und sonst $dreieck(X) = \sum_{i=1}^{X} i$. 

Der Funktor $quadrat$ hat die Stelligkeit 1. Es gilt $quadrat(0) = 0$ 
und sonst $quadrat(X) = \sum_{i=1}^{X} (2i-1)$.
\end{itemize}

Geben Sie den Wahrheitswert der folgenden Formeln an. Mit Begründung.
\begin{enumerate}
\item 
$\exists X : \bigl( \exists Y : \bigl( (\neg (X = Y)) \wedge (quadrat(X) = quadrat(Y)) \bigr) \bigr)$\\
\score{5}

\item 
$\forall X : \bigl( \exists Y : \bigl( (\neg (X = Y)) \wedge (quadrat(X) = quadrat(Y)) \bigr) \bigr)$\\
\score{5}

\item 
$\forall X : \bigl( \exists Y : 
\bigl(
(dreieck(X) + dreieck(Y)) = quadrat(X) 
\bigr)
\bigr)$\\
\score{5}

\item 
$\exists X: 
\Bigl( \forall Y : 
\Bigl(  \neg
\bigl(
(dreieck(X) + dreieck(Y)) = quadrat(X) 
\bigr)
\Bigr)
\Bigr)$


\end{enumerate}
\score{5}

\end{enumerate}



\end{aufgabe}
% Hier könnt ihr eure Lösung hinschreiben
\begin{loesung}

\end{loesung}
\end{document}
