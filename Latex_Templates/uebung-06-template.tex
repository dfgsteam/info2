% Dokoumentenklassen legen das grobe Layout fest. Hier wird eine selbsterstellte Klasse verwendet, deswegen benötigt man zum erstellen des PDFs auch die Datei uebungsblatt.cls und uebungsblatt.sty
\documentclass{uebungsblatt}

\usepackage[linesnumbered,commentsnumbered]{algorithm2e}
\usepackage{tikz}
\usepackage{hyperref}
% Store \nl in \oldnl
\let\oldnl\nl
% Remove line number for one line
\newcommand{\nonl}{\renewcommand{\nl}{\let\nl\oldnl}}

%Hier wird die Kopfzeile erstellt
\header{\textbf{Grundlagen der Praktischen Informatik}\hfill\textbf{Sommersemester 2024}\\
Student:in1 (Matrikelnummer1) % Name des Gruppenmitglieds Eintragen
\hfill Georg-August-Universität Göttingen \\
Student:in2 (Matrikelnummer2)  % Name des Gruppenmitglieds Eintragen
\hfill Institut für Informatik\\ 
Student:in3 (Matrikelnummer2)\\  % Name des Gruppenmitglieds Eintragen
Student:in4 (Matrikelnummer2)\\  % Name des Gruppenmitglieds Eintragen

% Hiermit wird der horizontale Strick erzeugt, der die Kopfzeile abgrenzt.
\rule{\textwidth}{0.1mm}}

% Hier wird die Blattnummer festgelegt.
\blattnummer{6}

% Hier beginnt der eigentliche Textkörper. Alles zwischen \begin{document} und \end{document} ist der eigentliche Text
\begin{document}

% Mit \underline kann man Sachen unterstreichen


% \begin{aufgabe} erstellt ein Aufgabenumgebung mit [...] könnt ihr den Titel angeben und mit \score die Punktzahl festlegen. Ist aber für euch nicht so wichtig.
\begin{aufgabe}[Nicht-rekursiver Parser I\score{48}]
\medskip 
Grammatik $G = (N, T, S, P\}$.

\begin{itemize}
\item[]
Nichtterminale
$N = \{$\textit{program, stmtList, stmt, expr, termTail, term, factorTail, factor, addOp, multOp}$\}$

Terminale $T = \{$\texttt{id, (, ), :=, +, -, *, /}$\}$

Startsymbol $S = $ \textit{program}

\begin{tabbing}
P = $\{\quad$ \= \textit{factorTail} \= \kill
Produktionen \\
P = $\{$ \> \textit{program}    \> $\rightarrow$ \textit{stmtList} \\
\> \textit{stmtList}   \> $\rightarrow$ \textit{stmt} \textit{stmtList} $|$ $\varepsilon$ \\
\> \textit{stmt}       \> $\rightarrow$ \texttt{id} \texttt{:=} \textit{expr} \\
\> \textit{expr}       \> $\rightarrow$ \textit{term} \textit{termTail} \\
\> \textit{termTail}   \> $\rightarrow$ \textit{addOp} \textit{term} \textit{termTail} $|$ $\varepsilon$ \\
\> \textit{term}       \> $\rightarrow$ \textit{factor} \textit{factorTail} \\
\> \textit{factorTail} \> $\rightarrow$ \textit{multOp} \textit{factor} \textit{factorTail} $|$ $\varepsilon$ \\
\> \textit{factor}     \> $\rightarrow$ \texttt{(} \textit{expr} \texttt{)} 
$|$ \texttt{id} \\
\> \textit{addOp}      \> $\rightarrow$ \texttt{+} $|$ \texttt{-} \\
\> \textit{multOp}     \> $\rightarrow$ \texttt{*} $|$ \texttt{/} \quad $\}$
\end{tabbing}
\end{itemize}

\begin{enumerate}
\item
Folgende FIRST-Mengen sind gegeben.

\medskip
\begin{small}	
FIRST(multOp factor factorTail) = $\{\texttt{*}, \texttt{/}\}$\\	
FIRST($\epsilon$) = \{$\epsilon$\}\\			
FIRST(\texttt{(} expr \texttt{)}) = FIRST(\texttt{(}) = $\{\texttt{(}\}$\\
FIRST(\texttt{id}) = \{\texttt{id}\}\\
FIRST(\texttt{+}) = \{\texttt{+}\}\\
FIRST(\texttt{-}) = \{\texttt{-}\}\\
FIRST(\texttt{*}) = \{\texttt{*}\}\\
FIRST(\texttt{/}) = \{\texttt{/}\}\\
\end{small}

Berechnen Sie für jede noch fehlende rechte Seite $\alpha$ der 
Produktion aus $P$ jeweils die Menge FIRST($\alpha$).\\
\score{18}

\item
Folgende FOLLOW-Mengen sind gegeben.

\medskip
\begin{small}
FOLLOW(program) = \{\$\$\}\\	
FOLLOW(stmtList) = \{\$\$\}\\
FOLLOW(stmt) = \{\texttt{id}, \$\$\}\\
FOLLOW(expr) = \{\texttt{id}, \texttt{)},\$\$\}\\
FOLLOW(multOp) = \{\texttt{(}, \texttt{id}\}
\end{small}

Berechnen Sie für die übrigen Nichtterminale $A$ die Menge FOLLOW($A$). \\
\score{18}
\item
Konstruieren Sie eine Parse-Tabelle mit Hilfe von FIRST und FOLLOW.\\
\score{12}

\smallskip
\underline{Hinweis.}
Es reicht aus in die Zellen die rechten Seiten der Produktionen einzutragen.

\end{enumerate}

\end{aufgabe}
% Hier könnt ihr eure Lösung hinschreiben
\begin{loesung} 

\end{loesung}
\newpage
%-----------------------------
\begin{aufgabe}[Nicht-rekursiver Parser II \score{19}]
\medskip
Betrachten Sie folgende Parse-Tabelle, \texttt{<1>} ist das Startsymbol
der zugehörigen Grammatik.

\bigskip
\begin{cut}
\markdown
\begin{lstlisting}[basicstyle=\ttfamily\footnotesize] \multicolumn{1}{l}{Stapel} 
|      | 0     | 1     | $neg$   | $vv$     | $^^$     | §§      |
| :--- | :---  | :---  | :---    | :---     | :---     | :---    |
| <1>  | <2>   | <2>   | <2>     |          |          |         |
| <2>  | <3><4>| <3><4>| <3><4>  |          |          |         |
| <3>  | 0     | 1     | $neg$<2>|          |          |         |
| <4>  |       |       |         |$vv$<3><4>|$^^$<3><4>|$epsilon$|
\end{lstlisting}

\medskip
\end{cut}
\begin{center}
            \begin{tabular}{l|l|l|l|l|l|l}
    		 & 0 & 1 & $\neg$ & $\lor$ & $\land$ & $\$\$$ \\
    		\hline
    		\verb|<1>| & \verb|<2>| & \verb|<2>| & \verb|<2>| & & & \\
    		\verb|<2>| & \verb|<3><4>| & \verb|<3><4>| & \verb|<3><4>| & & & \\
    		\verb|<3>| & $0$ & $1$ & $\neg$\verb|<2>| & & & \\
    		\verb|<4>| &  &  &  & $\lor$\verb|<3><4>| & $\land$\verb|<3><4>| & $\varepsilon$ \\
    	    \end{tabular}
\end{center}


\underline{Hinweis}

\smallskip
In den Zellen sind nur die rechten Seiten der Produktionen notiert.
Z.B. Zelle \verb|[<3>|, $\neg$ \verb|]| enthält $\neg$\verb|<2>|, 
d.h. der Inhalt dieser Zelle repäsentiert die Produktion 
\verb|<3>|$\rightarrow \neg$\verb|<2>|.


\begin{enumerate}
\item
Geben Sie die zugehörige Grammatik $G = (N, T, S, P\}$ an.\\
\score{7} 

\item

Stellen Sie dar wie ein nicht-rekursiver Parser unter Benutzung der 
Parse-Tabelle folgende Eingabe abarbeitet
($\$\$$ markiert das Ende der Eingabe).
\begin{align*}
\neg 0 \vee 0 \wedge 1 \vee \neg 1\ \$\$
\end{align*}

Vervollständigen Sie dazu nachfolgende Tabelle.\\
\score{12}

\underline{Hinweis}

In einer Tabellenzeile ist der Zustand des Stapels und die
noch nicht verarbeitete Eingaben, inklusive des aktuellen Eingabenzeichen,
dargestellt. Weiterhin die Aktion, die aus Stapel und Eingabe folgt.
Mögliche Aktionen sind z.B. Folgende.
\begin{itemize}
\item Anwendung einer Produktion, dann wird diese notiert.
\item Entfernen des obersten Stapelsymbols und des aktuellen Eingabensymbols, notiere \textit{match}.
\item Stapel $\gamma$, Eingabe $\$\$$, notiere \textit{stop}.
\item Ansonsten \textit{error}.
\end{itemize}

Aus der Aktion ergeben sich Stapel und Eingabe der nächsten Zeile.

\end{enumerate}

\bigskip
\begin{cut}
\markdown
\begin{lstlisting}[basicstyle=\ttfamily\footnotesize]
| Stapel           | Eingabe                    | Aktion         |
| ---:             | ---:                       | :---           | 
|<1>$gamma$        |$neg$0$vv$0$^^$1$vv$$neg$1§§|<1>$rarr$<2>    |
|<2>$gamma$        |$neg$0$vv$0$^^$1$vv$$neg$1§§|<2>$rarr$<3><4> |
|<3><4>$gamma$     |$neg$0$vv$0$^^$1$vv$$neg$1§§|<3>$rarr$$neg$<2>|
|$neg$<2><4>$gamma$|$neg$0$vv$0$^^$1$vv$$neg$1§§|match           |
|<2><4>$gamma$     |     0$vv$0$^^$1$vv$$neg$1§§|                |
\end{lstlisting}
\end{cut}

\begin{center}
            \begin{tabular}{l|r|l}
    		Stapel & \multicolumn{1}{l|}{Eingabe} & Aktion \\
    		\hline
    		\verb|<1>|$\gamma$ & $\neg 0 \lor 0 \land 1 \lor \neg 1 \$\$ $ & \verb|<1>| $\rightarrow$ \verb|<2>| \\
    		\verb|<2>|$\gamma$ & $\neg 0 \lor 0 \land 1 \lor \neg 1 \$\$ $ & \verb|<2>| $\rightarrow$ \verb|<3><4>|\\
    		\verb|<3><4>|$\gamma$ & $\neg 0 \lor 0 \land 1 \lor \neg 1 \$\$ $& \verb|<3>| $\rightarrow$ $\neg$\verb|<2>|\\
    		$\neg$\verb|<2><4>|$\gamma$ & $\neg 0 \lor 0 \land 1 \lor \neg 1 \$\$ $ & match\\
    	    \verb|<2><4>|$\gamma$ & $0 \lor 0 \land 1 \lor \neg 1 \$\$ $ & \\
    	    \end{tabular}
\end{center}
\end{aufgabe}
\begin{loesung}

\end{loesung}
\newpage
\begin{aufgabe}[LL-Syntaxanalyse \score{8}]
\begin{tabbing}
$P' = \{$ \= \textit{factorTail} \= \kill
$P' = \{$ \\
\> \textit{product}    \> $\rightarrow$ \textit{factor} \\
\> \textit{factor}     \> $\rightarrow$ \textit{factor} \texttt{*} \texttt{id} $|$ \texttt{id} \\
\> \textit{sum}        \> $\rightarrow$ \texttt{id} \textit{sumList} \\
\> \textit{sumList}    \> $\rightarrow$ \texttt{+} \textit{sum} $|$ \texttt{+} \texttt{(} \textit{product} \texttt{)} \\
$\}$
\end{tabbing}

Schreiben Sie die Produktionen in $P'$ so um, 
dass keine Linkrekursionen und keine 
gemeinsamen Präfixe mehr vorkommen.\\
\score{8}
\end{aufgabe}
% Hier könnt ihr eure Lösung hinschreiben
\begin{loesung}

\end{loesung}
\end{document}
