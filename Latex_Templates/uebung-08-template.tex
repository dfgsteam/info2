% Dokoumentenklassen legen das grobe Layout fest. Hier wird eine selbsterstellte Klasse verwendet, deswegen benötigt man zum erstellen des PDFs auch die Datei uebungsblatt.cls und uebungsblatt.sty
\documentclass{uebungsblatt}

\usepackage[linesnumbered,commentsnumbered]{algorithm2e}
\usepackage{tikz}
\usepackage{hyperref}
% Store \nl in \oldnl
\let\oldnl\nl
% Remove line number for one line
\newcommand{\nonl}{\renewcommand{\nl}{\let\nl\oldnl}}

%Hier wird die Kopfzeile erstellt
\header{\textbf{Grundlagen der Praktischen Informatik}\hfill\textbf{Sommersemester 2024}\\
Student:in1 (Matrikelnummer1) % Name des Gruppenmitglieds Eintragen
\hfill Georg-August-Universität Göttingen \\
Student:in2 (Matrikelnummer2)  % Name des Gruppenmitglieds Eintragen
\hfill Institut für Informatik\\ 
Student:in3 (Matrikelnummer2)\\  % Name des Gruppenmitglieds Eintragen
Student:in4 (Matrikelnummer2)\\  % Name des Gruppenmitglieds Eintragen

% Hiermit wird der horizontale Strick erzeugt, der die Kopfzeile abgrenzt.
\rule{\textwidth}{0.1mm}}

% Hier wird die Blattnummer festgelegt.
\blattnummer{8}

% Hier beginnt der eigentliche Textkörper. Alles zwischen \begin{document} und \end{document} ist der eigentliche Text
\begin{document}

% Mit \underline kann man Sachen unterstreichen


% \begin{aufgabe} erstellt ein Aufgabenumgebung mit [...] könnt ihr den Titel angeben und mit \score die Punktzahl festlegen. Ist aber für euch nicht so wichtig.
\begin{aufgabe}[Prädikatenlogik I \score{15}]
\medskip
Betrachten Sie folgende prädikatenlogische Formel.
\begin{align*}
\forall X:\Bigl(P(X)\implies\Bigl(\forall S:\Bigr(\exists R: \Bigr(\bigl(Q(f(R),T)\land Q(S,R)\bigr)\lor O(T)\Bigl)\Bigl)\Bigl)\Bigr)
\end{align*}
Gegeben ist eine Sprache über der Individuenmenge $D=[-10 .. 10]\subset \mathbb{Z}$
(also den ganzen Zahlen von $-10$ bis $10$ inklusive) mit folgender Signatur.

\smallskip
Die arithmetischen Operationen und die Größenrelationen über $D$ sind analog zu jenen
über den ganzen Zahlen $\mathbb{Z}$\ und werden als bekannt vorausgesetzt.
\begin{itemize}
    \item Prädikate $\mathcal{P}=\{O,P,Q\}$
    \begin{itemize}
        \item $O$ ist einstellig. $O(X)$ ist genau dann wahr, wenn $X=9$.
        \item $P$ ist einstellig. $P(X)$ ist genau dann wahr, wenn $X\leq 0$.
        \item $Q$ ist zweistellig. $Q(X,Y)$ ist genau dann wahr, wenn $X\geq Y$.
    \end{itemize}
    \item Funktoren $\mathcal{F} =\{f\}$
    \begin{itemize}
        \item $f$ ist einstellig. 
\begin{align*}
f(X) = \begin{cases}
\ 10 &\text{für } X = -10  \\
\ X-1 &\text{sonst}
\end{cases}
\end{align*}

    \end{itemize}
\end{itemize}
Bestimmen Sie die Menge $L(T)\subseteq D$ der Werte für $T$, mit denen die Formel den
Wahrheitswert \textbf{wahr} annimmt.\\
\score{15} 


\end{aufgabe}
% Hier könnt ihr eure Lösung hinschreiben
\begin{loesung} 

\end{loesung}
\newpage
%-----------------------------
\begin{aufgabe}[Prädikatenlogik II \score{15}]
\medskip
Betrachten Sie folgende prädikatenlogische Formel.
\begin{align*}
\forall S:\Bigl(\exists R:\Bigl(\bigl(Q(T,f(R))\lor Q(R,S)\bigr)\lor O(f(T))\Bigr)\Bigr)
\end{align*}
Gegeben ist eine Sprache über der Individuenmenge
\begin{align*}
D=\{\mathtt{Alice, Bob, Charlie, Diane}\}
\end{align*}
mit folgender Signatur.
\begin{itemize}
    \item Prädikate $\mathcal{P}=\{O,Q\}$
    \begin{itemize}
        \item $O$ ist einstellig. $O(X)$ ist genau dann wahr, wenn $X=\mathtt{Charlie}$.
        \item $Q$ ist zweistellig. $Q(X,Y)$ ist genau dann wahr, 
              wenn einer der folgenden Fälle eintritt.
        \begin{itemize}
            \item $(X,Y)=($\texttt{Bob, Alice}$)$
            \item $(X,Y)=($\texttt{Bob, Bob}$)$
            \item $(X,Y)=($\texttt{Alice, Charlie}$)$
        \end{itemize}
    \end{itemize}
    \item Funktoren $\mathcal{F}=\{f\}$
    \begin{itemize}
        \item $f$ ist einstellig. Es gilt Folgendes.
          \begin{itemize}
            \item $f(\mathtt{Alice})=\mathtt{Bob}$
            \item $f(\mathtt{Bob})=\mathtt{Alice}$
            \item $f(\mathtt{Charlie})=\mathtt{Diane}$
            \item $f(\mathtt{Diane})=\mathtt{Diane}$
          \end{itemize}
    \end{itemize}
\end{itemize}
Bestimmen Sie die Menge $L(T)\subseteq D$ der Werte für $T$, mit denen die Formel den
Wahrheitswert \textbf{wahr} annimmt. \\
\score{15} 

\end{aufgabe}
\begin{loesung}

\end{loesung}

\end{document}